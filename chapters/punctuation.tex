\chapter{Punctuation}





\section{A sentence end with punctuation}

\Cref{a mathematical text is a text} has an important consequence:
If a sentence ends with an equation or some kind of formula, then this need to be followed by some kind of punctuation (in most cases by a full point).

The following is a counterexample, which does it the wrong way:
\begin{tcblisting}{listing side text, title={Wrong}}
It follows that
\[
  a^2 + b^2 = c^2
\]
This formula is important.
\end{tcblisting}
The following is also wrong:
\begin{tcblisting}{listing side text, title={Wrong}}
It follows that:
\[
  a^2 + b^2 = c^2
\]
This formula is important.
\end{tcblisting}
The following example is right:
\begin{tcblisting}{listing side text, title={Right}}
It follows that
\[
  a^2 + b^2 = c^2.
\]
This formula is important.
\end{tcblisting}
It is even better if we add some slight spacing between the formula and the period.
\begin{tcblisting}{listing side text, title={Best}}
It follows that
\[
  a^2 + b^2 = c^2 \,.
\]
This formula is important.
\end{tcblisting}
This last approach is taken from \cite{tex_period}.
The following is very wrong:
\begin{tcblisting}{listing side text, title={Worst}}
It follows that
\[
  a^2 + b^2 = c^2
\]
.
This formula is important.
\end{tcblisting}






\section{Punctuation in commutative diagrams?}

There is some disagreement in the mathematical community about whether a commutative diagram is allowed to include punctuation coming from the surrounding text.
The author is of the opinion that a commutative diagram should not contain any such punctuation.
So when including a commutative diagram one has to either finish up the preceeding sentence beforehand, or has to incorporate the diagram in such a way that it contains no punctuation of the surround text.
\begin{tcblisting}{title={Finishing the sentence before the commutative diagram}}
  We consider the following commutative diagram:
  \[
    \begin{tikzcd}
        A
        \arrow{r}
        \arrow{d}
      &
        A'
        \arrow{d}
      \\
        C
        \arrow{r}
      &
        C'  
    \end{tikzcd}
  \]
  The horizontal arrows in this diagram are isomorphisms.
\end{tcblisting}
\begin{tcblisting}{title={Incorporating a commutative diagram in a sentence}}
  In the commutative diagram
    \[
    \begin{tikzcd}
        A
        \arrow{r}
        \arrow{d}
      &
        A'
        \arrow{d}
      \\
        C
        \arrow{r}
      &
        C'  
    \end{tikzcd}
  \]
  both horizontal arrows are isomorphisms.
\end{tcblisting}





\section{Use proper spacing after dots}

When writing abbreviations like \enquote{i.e.} or \enquote{e.g.} don’t simply write \texttt{i.e. words} or \texttt{e.g. words}.
\hologo{LaTeX} mistakes the trailing period by the end of sentence since it followed by a space and then makes this space larger then it should be.
This can be fixed by adding a backslash before the space, using \texttt{i.e.{\tbs} words} or \texttt{e.g.{\tbs} words}.
You can also use \texttt{i.e.{\customtexttilde}words} or \texttt{e.g.{\customtexttilde}words} if this space should not be broken.





\section{Lists contain punctuation}

Text that is organized using list environments still obeys the rules of punctuation.
Consider the following counterexample:
\begin{tcblisting}{listing side text, title = {Wrong}}
  A set $B$ is a basise of $V$ if
  \begin{enumerate}
    \item
      $B$ is linearly independent
    \item
      $B$ is a generating set
  \end{enumerate}
\end{tcblisting}
To figure out the correct punctuation simply remove the surrounding list and consider the resulting text.
In the above example this gives the following:
\begin{center}
  A set $B$ is a basis of $V$ if $B$ is linearly independent $B$ is a generating set
\end{center}
This is not a proper sentence, and should be the following instead:
\begin{center}
  A set $B$ is a basis of $V$ if $B$ is linearly independent and $B$ is a generating set.
\end{center}
The above counterexample should hence read as follows:
\begin{tcblisting}{listing side text, title = {Right}}
  A set $B$ is a basise of $V$ if
  \begin{enumerate}
    \item
      $B$ is linearly independent and
    \item
      $B$ is a generating set.
  \end{enumerate}
\end{tcblisting}
There are also some other acceptable versions:
\begin{tcblisting}{listing side text, title = {Right}}
  A set $B$ is a basise of $V$ if it satisfies the following conditions:
  \begin{enumerate}
    \item
      $B$ is linearly independent.
    \item
      $B$ is a generating set.
  \end{enumerate}
\end{tcblisting}




