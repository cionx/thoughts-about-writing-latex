\chapter{Mathematical layout}





\section{Environments for display mathematics}
\label{display environments}



\subsection{Don’t use \inlinecodetitle{\$\$  \$\$} or \envtitle{eqnarray}}

There are many good ways to put mathematical content into display mode, but \inlinecode{\$\$~\$\$}\massindex[display math environment]{\$\$~\$\$}[\inlinecode] and \envname{eqnarray}\massindex[display math environment]{eqnarray}[\envname] are none of them.
The method~\inlinecode{\$\$  \$\$} is too low level, and the environment~\envname{eqnarray} has too many problem and has long been deprecated.
One can instead use any of the following, depending on the planned usage.



\subsection{The environments~\inlinecodetitle{{\tbs}[ {\tbs}]} and \envtitle{equation*}}

The environments~\inlinecode{{\tbs}[ {\tbs}]}\massindex[display math environment]{{\tbs}[ {\tbs}]}[\inlinecode] and \envname{equation*}\massindex[display math environment]{equation*}[\envname] can be used for a single line of display math mode.
Both commands do the exactly same thing (when the package~\packname{amsmath}\massindex[packages]{amsmath}[\packname] is loaded).
\begin{showlatex}*{Using the environments~\inlinecode{{\tbs}[ {\tbs}]} and \envname{equation*}}
Suppose that both the formula
\[
  a + b = c
\]
and the formula
\begin{equation*}
  2a - b = c \,.
\end{equation*}
hold.
Then $a$ and $b$ are unique.
\end{showlatex}
The non-starred version~\envname{equation}\massindex[display math environment, tagging and numbering]{equation}[\envname] numbers the equation.
\begin{showlatex}*{Using the environment~\envname{equation}}
The formula
\begin{equation}
  a^2 - b^2 = (a + b)(a - b)
\end{equation}
is one of the binomial formulas.
\end{showlatex}



\subsection{The environment~\envtitle{gather*}}

The environment~\envname{gather*}\massindex[display math environment, multi-line mathematics]{gather*}[\envname] is meant for multiple lines that are non-aligned.
\begin{showlatex}{Using the environment~\envname{gather*}}
We consider for every integer $n \geq 0$ the polynomial
\[
  p_n
  =
  \sum_{k=0}^n x^k \,.
\]
In particular
\begin{gather*}
  p_0 = 1 \,,
  \qquad
  p_1 = 1 + x \,,
  \qquad
  p_2 = 1 + x + x^2 \,,
  \qquad
  p_3 = 1 + x + x^2 + x^3 \,,
  \\
  p_4 = 1 + x + x^2 + x^3 + x^4 \,,
  \qquad
  p_5 = 1 + x + x^2 + x^3 + x^4 + x^5 \,.
\end{gather*}
\end{showlatex}
The non-starred version~\envname{gather}\massindex[display math environment, multi-line mathematics, tagging and numbering]{gather}[\envname] numbers each line.
\begin{showlatex}{Using the environment~\envname{gather}}
We have the polynomials
\begin{gather}
  p_0 = 1 \,,
  \qquad
  p_1 = 1 + x \,,
  \qquad
  p_2 = 1 + x + x^2 \,,
  \qquad
  p_3 = 1 + x + x^2 + x^3 \,,
  \\
  p_4 = 1 + x + x^2 + x^3 + x^4 \,,
  \qquad
  p_5 = 1 + x + x^2 + x^3 + x^4 + x^5 \,.
\end{gather}
\end{showlatex}



\subsection{The environments~\envtitle{align*} and \envtitle{alignat*}}

The environment~\envname{align*}\massindex[display math environment, multi-line mathematics]{align*}[\envname] allows for multiple lines.
Each line contains the symbol~\inlinecode{\&}\massindex{\&}[\inlinecode] once, and the occurrences of this symbol are then aligned.
\begin{showlatex}*{Using the environment~\envname{align*}}
We find that
\begin{align*}
  a + b + c
  &=
  d + e + f + g
  \\
  &=
  h + i + j
  \\
  &=
  k + l + m + n \,.
\end{align*}
\end{showlatex}
The unstarred version~\envname{align}\massindex[display math environment, multi-line mathematics, tagging and numbering]{align}[\envname] numbers the lines.
\begin{showlatex}*{Using the environment \envname{align}}
We find again that
\begin{align}
  a + b + c
  &=
  d + e + f + g
  \\
  &=
  h + i + j
  \\
  &=
  k + l + m + n \,.
\end{align}
\end{showlatex}
One can also use multiple aligned columns, which then need to be separated by an additional~\inlinecode{\&}\massindex{\&}[\inlinecode].
For~$n$ aligned columns we hence need~$2n-1$ occurrences of~\inlinecode{\&} per line.
\begin{showlatex}{Using \envname{align*} with multiple columns}
We consider the values
\begin{align*}
  x_1 &= 1 \,,  &   x_2 &= 2 \,,  &   x_3 &= 3 \,,  \\
  x_4 &= 4 \,,  &   x_5 &= 5 \,,  &   x_6 &= 6 \,,  \\
  x_7 &= 7 \,,  &   x_8 &= 8 \,,  &   x_9 &= 9 \,.
\end{align*}
\end{showlatex}

The environment~\envname{alignat*}\massindex[display math environment, multi-line mathematics]{alignat*}[\envname] is similar to the environment~\envname{align*} but doesn’t add any built-in spacing between the aligned columns.
Any such spacing must therefore by added by hand.
One also has to specify the number of columns beforehand.
\begin{showlatex}{Using the environment~\envname{alignat*}}
We also consider the values
\begin{alignat*}{3}
  y_1 &= 9 \,,  &\qquad   y_2 &= 8 \,,  &\qquad   y_3 &= 7 \,,  \\
  y_4 &= 6 \,,  &         y_5 &= 5 \,,  &         y_6 &= 4 \,,  \\
  y_7 &= 3 \,,  &         y_8 &= 2 \,,  &         y_9 &= 1 \,.
\end{alignat*}
\end{showlatex}
To align multiple columns one should use \envname{alignat*} instead of \envname{align*} to get (manually) a good looking distance between the aligned columns.
The environment~\envname{alignat}\massindex[display math environment, multi-line mathematics]{alignat}[\envname] works in the same way as \envname{alignat*} but automatically numbers the lines.



\subsection{Don’t use \envtitle{center} plus \inlinecodetitle{\$ \$}}
\massindex{center}[\envname]
\massindex{\$ \$}[\inlinecode]

By all that is holy, don’t do the following:
\begin{showlatex}*{Using~\comname{center} with~\inlinecode{\$ \$}}
It holds that
\begin{center}
  $\sum_{i=0}^n 2^i = 2^{n+1} - 1$.
\end{center}
This can be shown by induction.
\end{showlatex}
% TODO: Explain all the things that go wrong



\subsection{The environments~\envtitle{gathered}, \envtitle{aligned} and \envtitle{alignedat}}

The environments~\envname{gathered}\massindex[multi-line mathematics]{gathered}[\envname], \envname{aligned}\massindex[multi-line mathematics]{aligned}[\envname] and \envname{alignedat}\massindex[multi-line mathematics]{alignedat}[\envname] are variations of the environments~\envname{gather*}, \envname{align*} and \envname{alignat*} that can be used inside an already existing math environment.
\begin{showlatex}*{Using \envname{aligned}}
\[
  \left\{
    \begin{aligned}
      a + b &= c      \\
      d     &= e + f
    \end{aligned}
  \right\}
\]
\end{showlatex}



\subsection{Overview}

One should always use the most basic environment that does the job:
Using overpowered environments can lead to unexpected problems.
Consider the following example:
\begin{showlatex}{Improper use of~\envname{align*}}
\begin{align*}
  ABCD = EF = GHI = JKL = M = NOP
  \\
  QRS = TUV = WX = Y = Z
\end{align*}
\end{showlatex}
The environment~\envname{align*} automatically alignes both lines on the right since no information about alignment was given.
In the above situation one should use \envname{gather*} instead.
\begin{showlatex}{Using \envname{gather*} for non-aligned equations}
\begin{gather*}
  ABCD = EF = GHI = JKL = M = NOP
  \\
  QRS = TUV = WX = Y = Z
\end{gather*}
\end{showlatex}
If a multi-line display mode environment is used for a single line then one can get spacing issues, see \cref{spacing before multi-line}.

The flowchart in \cref{environment flow chart} explains how to choose the correct display environment.
(This flowchart is partly inspired by \cite{flowchart}.)
\begin{figure}[tb]
  \begin{center}
  \begin{tikzpicture}[
    node distance = 5em,
    >={Latex[width=2mm,length=2mm]},
    every text node part/.style = { align = center },
    start/.style    = { rectangle,
                        rounded corners,
                        fill = green!20!white,
                        draw = black,
                        minimum width  = 5em,
                        minimum height = 2em
                      },
    question/.style = { rectangle,
                        rounded corners,
                        fill = black!10!white,
                        draw = black,
                        minimum width  = 9em,
                        minimum height = 2.5em
                      },
    answer/.style   = { rectangle,
                        rounded corners,
                        draw = black,
                        minimum width  = 7em,
                        minimum height = 1.5em
                      }
  ]
    % nodes
    \node (start)
          [start]
          {formula};
    \node (multi-line)
          [question, below of = start]
          {multi-line?};
    \node (alignment)
          [question, below of = multi-line]
          {alignment?};
    \node (multicolumn)
          [question, below of = alignment]
          {multiple columns?};
    \node (equation)
          [answer, right of = multi-line, xshift = 7em]
          {\inlinecode{{\tbs}[ {\tbs}]}\\\envname{equation*}};
    \node (gather)
          [answer, right of = alignment, xshift = 7em]
          {\inlinecode{gather*}};
    \node (align)
          [answer, right of = multicolumn, xshift = 7em]
          {\inlinecode{align*}};
    \node (alignat)
          [answer, below of = align]
          {\inlinecode{alignat*}};
    % arrows
    \draw[->] (start) -- (multi-line);
    \draw[->] (multi-line)   -- node[anchor=south] {no} (equation);
    \draw[->] (multi-line)   -- node[anchor=east] {yes} (alignment);
    \draw[->] (alignment)   -- node[anchor=south] {no} (gather);
    \draw[->] (alignment)   -- node[anchor=east] {yes} (multicolumn);
    \draw[->] (multicolumn) -- node[anchor=south] {no} (align);
    \draw[->] (multicolumn) |- node[anchor=east, yshift=1.7em] {yes} (alignat);
  \end{tikzpicture}
  \end{center}
  If numbering of the line(s) is needed then the unstarred version is to be used.
  \caption{Deciding on a math environment.}
  \label{environment flow chart}
\end{figure}





\section{Where to break and align formulas}
\label{break and align ponts}

Oftentimes a formula is broken among multiple lines.
This is done for at least two reasons:
\begin{myitemize}
  \item
    To improve the readability of the given formula.
  \item
    To prevent that the formula goes over the margins of the text area.
\end{myitemize}
If a formula is broken among multiple lines then one has to choose at which places the formula should be broken, and how the resulting parts of the formula will then be aligned.
In this section we present some standard ways of doing so.



\subsection{When to break a formula}
\index{line breaks!in math mode|(}

We first discuss when a formula needs to be broken.

\subsubsection{When the formula is too long}

If a formula is too long to physically fit into the text area, i.e.\ if it goes over the margins of the text area, then it must be broken up.
These occurrences are easy to spot since they give will give an \enquote{overfull hbox} warning.
Consider the following example:
\begingroup
\begin{showlatex}[before lower = {\hfuzz = 40pt}, after lower = {\hfuzz = 0pt}]{An overfull hbox}
  It follows from
  \[
    aaaaaaa
    =
    bbbbbbb
    =
    ccccccc
    \leq
    ddddddd
    =
    eeeeeee
    =
    fffffff
    \leq
    ggggggg
    =
    hhhhhhh
  \]
  that $a \leq h$.
\end{showlatex}
\endgroup

\subsubsection{When the formula is visually too long}

Sometimes a formula does fit into a single line, but barely so.
Consider the following example:
\begin{showlatex}{Visually overfull hbox}
Here is some text.
\[
  aaaaaaaaa = bbbbbbbbbbbbbb = cccccccccccccc = dddddddddddddd = eeeeeeeeee
\]
Here is some more text.
\end{showlatex}
The formula in the above example does -- technically speaking -- not go over the margin.
But it has stretched the display mode beyond its visual limits.
The formula does therefore need to be broken up.

\subsubsection{Expressing structure}

Often the breaking up of a formula is done to better express the structure -- and thus content -- of the displayed formula.
Consider the following example:
\begin{showlatex}[label={unreadable formula}]{A formula that should be broken up for readability}
If $k$ is algebraically closed with~$\operatorname{char}(k) \neq 2$ and $i$ is a square root of $-1$ then
\[
  k[x]/(x^2 + 1)
  =
  k[x]/( (x - i) (x + i) )
  \cong
  k[x]/(x - i) \times k[x]/(x + i)
  \cong
  k \times k
\]
by the Chinese remainder theorem.
\end{showlatex}
The above output still has an appropriate length to be put into a single line, and if space is sparse then this is an acceptable solution.
But this single-line approach to the formula does not help to display its internal structure.
This can be done by splitting up the formula as done in the next example:
\begin{showlatex}{Broken up version of \cref*{unreadable formula}}
If $k$ is algebraically closed and $i$ is a square root of $-1$ then
\begin{align*}
  k[x]/(x^2 + 1)
  &=
  k[x]/( (x - i) (x + i) )
  \\
  &\cong
  k[x]/(x - i) \times k[x]/(x + i)
  \\
  &\cong
  k \times k
\end{align*}
by the Chinese remainder theorem.
\end{showlatex}
This form makes it clear where the equalities and isomorphisms occur.



\subsection{Where to break and align a formula}
\index{aligning formulas|(}

We now discuss at which points a formula can be broken, and how these broken parts can then be aligned.

\subsubsection{Aligning at relation symbols I}

A first approach is to put all relation symbols underneath each other.
\begin{showlatex}{Aligning relation symbols~I}
Let $x$, $y$ be two commuting, nilpotent elements of~$A$.
Then
\begin{align*}
  \exp(x) \exp(y)
  &=
  \left( \sum_{k=0}^\infty \frac{x^k}{k!} \right)
  \left( \sum_{l=0}^\infty \frac{y^l}{l!} \right)
  \\
  &=
  \sum_{k,l=0}^\infty \frac{x^k y^l}{k! \, l!}
  \\
  &=
  \sum_{n=0}^\infty \, \sum_{k+l = n} \frac{x^k y^l}{k! \, l!}
  \\
  &=
  \sum_{n=0}^\infty \frac{1}{n!} \sum_{k=0}^n \binom{n}{k} x^k y^{n-k}
  \\
  &=
  \sum_{n=0}^\infty \frac{1}{n!} (x + y)^n
  \\
  &=
  \exp(x + y) \,.
\end{align*}
\end{showlatex}

\subsubsection{Aligning at relation symbols II}

One can also align all relation symbols to the left, so that the broken up parts of the formula all lie on top of each other.
\begin{showlatex}{Aligning relation symbols~II}
Let~$R$ and~$S$ be two commutative rings.
Then for every~$(r,s) \in R \times S$,
\begin{align*}
  {}&
  (r, s) \in (R \times S)^\times
  \\
  \iff{}&
  \text{there exist $(r', s') \in R \times S$ with $(r,s)(r',s') = (1,1)$}
  \\
  \iff{}&
  \text{there exist $r' \in R$ and $s' \in S$ with $rr' = 1$ and $ss' = 1$}
  \\
  \iff{}&
  \text{$r \in R^\times$ and $s \in S^\times$}
  \\
  \iff{}&
  (r,s) \in R^\times \times S^\times \,,
\end{align*}
and therefore $(R \times S)^{\times} = R^\times \times S^\times$.
\end{showlatex}
The empty pair of curly brackets in \inlinecode{{\tbs}iff\{\}\&} ensures that the spacings coming from \comname{iff} and~\inlinecode{\&} do not interfere with each other.
Otherwise something like this happens:
\begin{showlatex}*{Wrong spacing when alignment points are set wrong}
  \begin{align*}
     &\text{some stuff} \\
    =&\text{some other stuff}
  \end{align*}
\end{showlatex}

\subsubsection{Breaking at a binary operator}

Sometimes it is also useful to break a long term of a formula at a binary operator.
In this case this operator needs to occur in the line after the break.
The following example does it wrong:
\begin{showlatex}*{Wrong aligning at a binary operator~I}
\begin{align*}
  & a + b + c + \\
  & d + e
\end{align*}
\end{showlatex}
The following should be done instead:
\begin{showlatex}*{Right breaking at a binary operator~I}
\begin{align*}
  & a + b + c \\
  & + d + e
\end{align*}
\end{showlatex}
%TODO: Explain the above example better.
If a formula is broken at relation symbols and one of the resulting terms is broken at a binary operator, then the operator is not aligned together with the relation symbols.
Instead the binary operator appears after the relation symbols.
The following example does it wrong:
\begin{showlatex}*{Wrong breaking at a binary operator~II}
\begin{align*}
  a + a
  &=
  b + b + b + b
  \\
  &+
  b + b + b + b
  \\
  &=
  c + c + c + c + c
\end{align*}
\end{showlatex}
Instead the following has to be done:
\begin{showlatex}*{Right breaking at a binary operator~II}
\begin{align*}
  a + a
  ={}&
  b + b + b + b
  \\
  {}&
  +b + b + b + b
  \\
  ={}&
  c + c + c + c + c
\end{align*}
\end{showlatex}

\subsubsection{A single term in another line}

If a single line equation is too long, then it is sometimes appropriate to put the last term in a new line, such that the last term occurs below the second to last term.
\begin{showlatex}{Single term in new line}
It follows that
\begin{align*}
  aaaaaaaaaaaa
  =
  bbbbbbbbbbb
  =
  cccccccccc
  =
  ddddddddd
  &=
  eeeeeeee
  \\
  &=
  ffffff
\end{align*}
and hence $2 + 2 = 5$.
\end{showlatex}

\index{aligning formulas|)}
\index{aligning formulas!zzzza@\igobble |seealso {\comname{align*}}}
\index{aligning formulas!zzzzb@\igobble |seealso {\comname{alignat*}}}



\subsection{Don’t break formulas badly}

One should always keep in mind that breaking up a formula isn’t just meant to prevent technical problems, but more importantly to let the resulting output better display the structure -- and thus part of the content -- of the formula.
A badly broken up formula is harder to understand for both the reader and the author.
Consider the following example:
\begin{showlatex}{Badly broken formula}
It follows that
\begin{align*}
  aaaaaaaaa
  &=
  bbbbbbbbbbb
  =
  cccccccc
  \leq
  dddddd
  =
  eeeeeee
  \\
  &=
  eee
  \leq
  fffffff
  =
  gggggggg
  \leq
  hhhhhhhhh
  \\
  &<
  kkkkkkk
  =
  llll
  \leq
  mmmma
  =
  nnnnnnn
  =
  pp \,.
\end{align*}
\end{showlatex}
In such a case the breaking of the formula should be done in a consistent way.
There seem to be two sensible approaches for the above example, which we will now explain.

\subsubsection{Align everything}

One can align all occurring relation symbols:
\begin{showlatex}{Aligning all relation symbols}
It follows that
\begin{align*}
  aaaaaaaaa
  &= bbbbbbbbbbb \\
  &= cccccccc \\
  &\leq dddddd \\
  &= eeeeeee \\
  &= eee \\
  &\leq fffffff \\
  &= gggggggg \\
  &\leq hhhhhhhhh \\
  &< kkkkkkk \\
  &= llll \\
  &\leq mmmma \\
  &= nnnnnnn \\
  &= pp \,.
\end{align*}
\end{showlatex}
This approach has the advantage of being very consistent.
But it has the disadvantage of taking a lot of space.
It might also not reflect the structure of the formula particularly well, as this layout gives all (in)equalities the same importance.
    
\subsubsection{Align at inequalities}

On could align all the inequality symbols, to make it clear where these occur:
\begin{showlatex}{Aligning all inequalities}
It follows that
\begin{align*}
  aaaaaaaaa
  &= bbbbbbbbbbb
  = cccccccc
  \\
  &\leq
  dddddd
  = eeeeeee
  = eee
  \\
  &\leq
  fffffff
  = gggggggg
  \\
  &\leq
  hhhhhhhhh
  \\
  &<
  kkkkkkk
  = llll \\
  &\leq mmmma
  = nnnnnnn
  \\
  &= pp \,.
\end{align*}
\end{showlatex}
This layout emphasizes the importance of the inequalities, while relegating the equalities to a less important position.
Note that we have also aligned the first and last equality signs to make it clear where the manipulations begin and end.
If some other equalities are also particularly important (e.g.\ if they follows from some previously hard-earned proposition) then they too should be aligned

\index{line breaks!in math mode|)}





\section{Aligning nearly aligned formulas}
\index{aligning formulas|(}

Sometimes formulas turn out to look nearly aligned in the compiled output, even though this wasn’t planned.
But the formulas may still be non-aligned enough to look jarring.
In such a case it is often best to align these formulas.

Consider the following example:
\begin{showlatex}{Accidental jarringly non-aligned expressions}
\begin{gather*}
  KK^{-1} = 1 = K^{-1}K \,,
  \quad
  EF - FE = \frac{ K - K^{-1} }{ q - q^{-1} } \,,
  \\
  KE = q^2 EK \,,
  \quad
  KF = q^{-2} FK \,.
\end{gather*}
\end{showlatex}
Note that the first line of the output looks slightly shifted to the left when compared to the second line.
This impression vanishes when both lines are properly aligned, and instead gives rise to more coherent look and fell.
\begin{showlatex}{Intentional well-aligned expressions}
\begin{align*}
  KK^{-1} = 1 = K^{-1}K \,,
  \quad
  &EF - FE = \frac{ K - K^{-1} }{ q - q^{-1} } \,,
  \\
  KE = q^2 EK \,,
  \quad
  &KF = q^{-2} FK \,.
\end{align*}
\end{showlatex}

\index{aligning formulas|)}





\section{Proper spacing before multi-line display mode environments}
\label{spacing before multi-line}

If mathematical content is put into display mode then this content will not only be horizontally centered but also receive some vertical spacing\index{spacing!around display mode} around it to separate it from the surrounding text.
Observe in the following example that both above and below the formula there is an additional spacing of roughly (exactly?) one line:
\begin{showlatex}*{Vertical space around display mode}
Lorem ipsum dolor sit amet, consectetur adipiscing elit, sed do eiusmod tempor incididunt ut labore et
\[
  a = b \,.
\]
Nunc aliquet bibendum enim facilisis gravida. Nisl nunc mi ipsum faucibus vitae aliquet nec ullamcorper.
\end{showlatex}

If the line before a display mode environment is too short then this vertical spacing may be visually too large.
The one-line display environments~\inlinecode{{\tbs}[ {\tbs}]}\massindex[display math environment]{{\tbs}[ {\tbs}]}[\inlinecode] and~\envname{equation*}\massindex[display math environment]{equation*}[\envname] do therefore automatically adjust this spacing.
Observe that in the following example the distance between the first two text lines is far shorter than the distance between the second and third text lines.
\begin{showlatex}*{Adjusted spacing before one-line display mode}
Lorem
\[
  a = b
\]
 ipsum dolor sit amet, consectetur
\[
  a = b \,.
\]
ipsum elit.
\end{showlatex}
We can say more precisely that the additional spacing inserted by {\LaTeX} is either \comname{abovedisplayskip} or its shorter version~\comname{abovedisplayshortskip}.

The multi-line display mode environments (like~~\envname{align*}\massindex[display math environment, multi-line mathematics]{align*}[\envname]) don’t have this feature (for technical reasons).
Instead, it always uses the spacing~\comname{abovedisplayskip}.
This can lead to some unpleasant spacing:
\begin{showlatex}*{Unpleasant spacing around a multi-line environment}
text text text text text text text text text
\begin{alignat*}{1}
  x &= y_i
\shortintertext{text}
  x &= z_{i+1}
\end{alignat*}
text text text text text text text text
\end{showlatex}

The package~\packname{mathools} provides a partial fix to this problem:
By putting the command~\comname{SwapAboveDisplaySkip}\massindex[\piname{mathtools}, spacing!around display mode]{SwapAboveDisplaySkip}[\comname] at the beginning of a multi-line display mode environment we force {\LaTeX} to use the shorter spacing~\comname{abovedisplayshortskip}.
\begin{showlatex}*{Pleasant spacing around a multi-line environment}
text text text text text text text text text
\begin{alignat*}{1}
\SwapAboveDisplaySkip
  x &= y_i
\shortintertext{text}
  x &= z_{i+1}
\end{alignat*}
text text text text text text text text
\end{showlatex}





\section{Use the \envtitle{cases} environment}

Use the environment~\envname{cases}\massindex[multi-line mathematics]{cases}[\envname] for case distinctions:
\begin{showlatex}{Using \envname{cases}}
It follows that
\[
  A(x)
  =
  \begin{cases}
    x^2  & \text{if $x \leq 0$,} \\
    3x   & \text{if $x = 0$.}
  \end{cases}
\]
\end{showlatex}
In most cases one should actually use the environment~\envname{cases*}\massindex[multi-line mathematics]{cases*}[\envname], which ensures that the second column will be treated as text.
\begin{showlatex}{Using \envname{cases*}}
It follows that
\[
  A(x)
  =
  \begin{cases*}
    x^2  & if $x \leq 0$, \\
    3x   & if $x = 0$.
  \end{cases*}
\]
\end{showlatex}

The package~\packname{mathtools}\massindex[packages]{mathtools}[\packname] defines some more useful variants of the environment~\envname{cases}.
We refer to~\cite[3.4.3]{mathtools} for more information on this.




\section{Proper placement of the qed-symbol}

\index{qed-symbol|(}

The~\envname{proof}~environment\massindex[\piname{amsthm}]{proof}[\envname] automatically places a qed-symbol at its end.
\begin{showlatex}{Using the \envname{proof}~environment}
\begin{proof}
This is obviously a proof.
\end{proof}
\end{showlatex}
It can happen that this automatic placement of the qed-symbol gives a bad looking result.
The general rule is that the qed-symbol should not occur at the end of an otherwise empty line.
To fix such a bad placement one can use the command~\comname{qedhere}\massindex{qedhere}[\comname].
Let’s look at some specific examples of this problem:

\subsubsection{qed-symbol after lists}

If a proof consists of a list then the qed-symbol will by default be placed after the list, and thus in a new line.
\begin{showlatex}{Improper placement of the qed-symbol after a list~I}
\begin{proof}
  This proof consists of a list.
  \begin{enumerate}
    \item
      Some part of the proof.
    \item
      Another part of the proof.
  \end{enumerate}
\end{proof}
\end{showlatex}
This can be fixed by placing the command~\comname{qedhere} just before the list ends.
\begin{showlatex}{Proper placement of the qed-symbol after a list}
\begin{proof}
  This proof consists of a list.
  \begin{enumerate}
    \item
      Some part of the proof.
    \item
      Another part of the proof.
    \qedhere
  \end{enumerate}
\end{proof}
\end{showlatex}
But one has to be careful no to introduce a new line before the command~\comname{qedhere}.
\begin{showlatex}{Improper placement of the qed-symbol after a list~II}
\begin{proof}
  This proof consists of a list.
  \begin{enumerate}
    \item
      Some part of the proof.
    \item
      Another part of the proof.
      
    \qedhere
  \end{enumerate}
\end{proof}
\end{showlatex}

\subsubsection{qed-symbol after displaystyle}

If a proof ends with display~style mathematics then the qed-symbol will by default be placed after this display~mode.
\begin{showlatex}{Improper placement of the qed-symbol after display~mode~I}
\begin{proof}
  This proof ends with a some display~mode mathematics, namely
  \[
    \sin(x+y)
    =
    \cos(x)\sin(y) + \sin(x)\cos(y) \,.
  \]
\end{proof}
\end{showlatex}
There are at least two different approaches to fixing this situation:

Some authors prever to bring the qed-symbol to the height of the display environment, as done in the following example.
\begin{showlatex}{Somewhat proper placement of the qed-symbol after display~mode}
\begin{proof}
  This proof ends with a some display~mode mathematics, namely
  \[
    \sin(x+y)
    =
    \cos(x)\sin(y) + \sin(x)\cos(y) \,.
    \qedhere
  \]
\end{proof}
\end{showlatex}
But this approach does not work if the (last line) of the display environment has additional height.
Then the mathematical formula goes below the line on which the qed-symbol rests.
\begin{showlatex}{Improper placement of the qed-symbol after display~mode~II}
\begin{proof}
  This proof ends with a some display~mode mathematics, namely
  \[
    1
    + \frac{a}{b - a}
    =
    \frac{(b - a) + a}{b - a}
    =
    \frac{b}{b - a} \,.
    \qedhere
  \]
\end{proof}
\end{showlatex}
The author is the opinion that in such a cases there is no good placement for the qed-symbol.

Aside the placement of the qed-symbol there is another problem to ending a proof with a display~environment:
One of the functions of display~mode is to put an emphasis on the displayed mathematical content.
Putting such an emphasis directly before the end of a proof can lead to the proof missing a sense of closure.

This problem leads to the second solution for the placement of the qed-symbol.
Never end a proof with a display~environment.
This can be done by rewriting the end of proof, or by adding a closing sentence to it.
% TODO: Give an example

\subsubsection{qed-symbol at the end of a long text}

It can also happen that the qed-symbol is pushed to a new line if the previous line is completely filled with text.
(Although {\LaTeX} will actually try quite hard to prevent this from happening.)
If this happens then one should (slightly) rewrite the text to circumvent this problem.

\index{qed-symbol|)}
\index{qed-symbol!zzzz@\igobble |seealso{\comname{qedhere}}}




\section{Tagging and numbering}

A finer control over tags can be achieved via the commands~\comname{tag} and~\comname{notag}.



\subsection{Don’t autonumber all formulas}
\label{dont number all formulas}
\index{tagging and numbering}

Don’t indiscriminately number every occurring formula.
Instead, an equation should be numbered only if it will be referred to later on.
This numbering should then be done automatically by using a suitable environment like~\envname{equation},~\envname{gather},~\envname{align} or~\envname{alignat}.



\subsection{\comtitle{tag}}

With the command~\comname{tag}\massindex[tagging and numbering]{tag}[\comname] a custom tag can be set.
This is useful for marking selected equations by special symbols:
\begin{showlatex}{Using~\comname{tag} for marking a line}
Consider the equation
\begin{equation}
\label{important equation}
  2 + 2 = 5 \,.
  \tag{\ast}
\end{equation}
Note that \cref{important equation} can equivalently be expressed as~$5 = 2 + 2$.
\end{showlatex}
The argument of~\comname{tag} is in text mode, and the resulting tag is automatically enclosed in parentheses.
These parentheses can be removed by using the starred command~\comname{tag*} instead.

The command~\comname{tag} should not be used for regular numbering of equations.
It should be used to tag only certain (often a single) equations in a special way.

It can also be used to express that certain transformations have been used, as the following example demonstrates:
\begin{showlatex}{Using~\comname{tag} to explain steps}
It follows from the Chinese remainder theorem that
\begin{align*}
  \mathbb{R}[x] / ( x^3 + x^2 + x + 1 )
  &=
  \mathbb{R}[x] / ( (x^2 + 1) (x + 1) )
  \\
  &\cong
  \mathbb{R}[x] / ( x^2 + 1 ) \times \mathbb{R}[x] / ( x + 1 )
  \tag{CRT}
  \\
  &\cong
  \mathbb{C} \times \mathbb{R}
\end{align*}
\end{showlatex}



\subsection{\comtitle{notag}}

According to \cref{dont number all formulas} a formula should be numbered only if it needs to be referred to.
But if this formula occurs in a multi-line environment like~\envname{align} then all occuring lines will be numbered, even unwanted ones.
The prevent the numbering of the unrequired lines the command~\comname{notag}\massindex[tagging and numbering]{tag}[\comname] can then be used:
\begin{showlatex}*{Using~\comname{notag} to prevent selected line numbers}
\begin{align}
  a
  &= b \notag \\
  &= c \\
  &= d \notag \\
  &= e
\end{align}
\end{showlatex}





\section{Multi-line set descriptions}

Multi-line set descriptions of the form
\[
  \left\{
    (e_1, \dotsc, e_n)
  \,\middle|\,
    \begin{tabular}{@{}c@{}}
      $e_1, \dotsc, e_n \in R$ \\
      is a complete set of \\
      pairwise orthogonal \\
      idempotents
    \end{tabular}
  \right\}
\]
can be typeset by using a \envname{tabular} environment for the right hand side of the set description:
\begin{showlatex}*{Multi-line set descriptions with tabular}
\[
  \left\{
    x \in X
  \,\middle|\,
    \begin{tabular}{@{}c@{}}
      $x$ satisfies \\
      certain conditions
    \end{tabular}
  \right\}
\]
\end{showlatex}
Note that the entries of the environment~\envname{tabular} are automatically in text mode.
The argument~\inlinecode{\@\{\}} ensure that the environment~\envname{tabular} does not insert additional spacing\index{spacing!in math mode} to its left and right.




