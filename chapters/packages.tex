\chapter{Useful packages}





\section{Package: \texttt{microtype}}

Use the \texttt{microtype} package.
It makes your document look nicer and can help you circumvent overfull hboxes.
Simply including the package is enough to let it work its magic.





\section{Package: \texttt{mathtools}}

The \texttt{amsmath} package is standard for mathematical typesetting.
The \texttt{mathtools} package is an extension of the \texttt{amsmath} package that fixes some of its problems and also provides some new (and often times very useful) functionalities.
The \texttt{mathtools} package automatically loads the \texttt{amsmath} package too, so instead \commandtt{usepackage\{amsmath\}} just use \commandtt{usepackage\{mathtools\}}.





\section{Package: \texttt{csquotes}}

Dealing with quotations marks by hand can be a pain in the ass, for at least two reasons:
Different languages use different kinds of quotation marks, and finding the right combination of \hologo{LaTeX} code to get the correct ones can be a non-trivial problem.
One way to circumvent this probem is by using the \texttt{csquotes} package, which provides the command \texttt{enquote}:
\begin{tcblisting}{listing side text, title={\commandtt{enquote} for quotes}}
\enquote{This is a quote.}
\end{tcblisting}
When a different languge is loaded using \texttt{babel}, then by loading the \texttt{csquotes} package with the option \texttt{babel=true} ensures the correct quotation marks for the specified language:
\begin{tcblisting}{listing side text, title={\texttt{csquotes} chooses the right kind of quotation marks}}
% american english
\selectlanguage{american}
\enquote{quote}
% british english
\selectlanguage{british}
\enquote{quotation}
% german
\selectlanguage{ngerman}
\enquote{Zitat}
% french
\selectlanguage{french}
\enquote{citation}
\end{tcblisting}

The \texttt{csquotes} command automatically handels nested quotation marks:
\begin{tcblisting}{listing side text, title={Nested quotes}}
\enquote{This is a \enquote{quote} inside a quote.}
\end{tcblisting}
So for dealing with quotes of any kind use the \texttt{csquotes} package.





\section{Package: \texttt{enumitem}}

\hologo{LaTeX} provides three different kinds of list environments:
Numbered lists are provided by the \texttt{enumerate} environment.

\begin{tcblisting}{listing side text, title={\texttt{enumerate} environment}}
\begin{enumerate}
  \item
    Assumption
  \item
    ???
  \item
    Contradiction
\end{enumerate}
\end{tcblisting}
Unnumbered lists are provided by the \texttt{itemize} environment.
\begin{tcblisting}{listing side text, title={\texttt{itemize} environment}}
\begin{itemize}
  \item
    This is a list item.
  \item
    This is also a list item.
  \item
    And yet another list item.
\end{itemize}
\end{tcblisting}
The \texttt{decription} environment uses no predefined symbols for the list items and instead expects a descriptive text from the author.
\begin{tcblisting}{listing side text, title={\texttt{description} environment}}
\begin{description}
  \item[Field]
    A ring in which every nonzero element has an inverse.
  \item[Vector space]
    A module over a field.
  \item[Ring]
    A ring without inverses.
  \item[Module]
    A vector space over a ring.
\end{description}
\end{tcblisting}

The package \texttt{enumitem} is immensely usefull to configure the style and behavior of these list environments.
It provides (among others) the following features:
\begin{itemize}
  \item
    For \texttt{enumerate} environment the style of the numbering can be changed.
    
    \begin{tcblisting}{listing side text, title={Changing numbering style}}
\begin{enumerate}[label = (\alph*)]
  \item
    First entry.
  \item
    Second entry.
  \item
    Third entry.
\end{enumerate}
    \end{tcblisting}

  One has the following options:
    \begin{center}
      \begin{tabular}{@{}ll@{}}
        \toprule
        option
        &
        output
        \\
        \midrule
        \commandtt{alph*}
        &
        lower case alphabetic
        \\
        \commandtt{Alph*}
        &
        upper case alphabetic
        \\
        \commandtt{roman*}
        &
        lower case roman
        \\
        \commandtt{Roman*}
        &
        upper case roman
        \\
        \commandtt{arabic*}
        &
        arabic numbers
        \\
        \bottomrule
      \end{tabular}
    \end{center}
    
    One can similarly change the symbol for \texttt{itemize} lists:
    \begin{tcblisting}{listing side text, title={Changing \texttt{itemize} symbol}}
\begin{itemize}[label = {\textbullet}]
  \item
    First entry.
  \item
    Second entry.
\end{itemize}
Now with another symbol:
\begin{itemize}[label = {\textopenbullet}]
  \item
    First entry again.
  \item
    Second entry again.
\end{itemize}
    \end{tcblisting}
    
  \item
    One can resume lists:
    \begin{tcblisting}{title={Resuming lists}}
Some text before the first \texttt{enumerate} environment.
\begin{enumerate}
  \item
    First entry.
  \item
    Second entry.
\end{enumerate}
Some text between the \texttt{enumerate} environment.
\begin{enumerate}[resume]
  \item
    Third entry.
  \item
    Fourth entry.
\end{enumerate}
Some text after the second \texttt{enumerate} environment.
    \end{tcblisting}
    
  \item
    One can change the various spacings involved in the list environments.
%   TODO: Gixe examples

  \item
    Gobal settings can be set:
    \begin{tcblisting}{listing side text, title={Global settings}}
\setlist[enumerate]{label = \roman*)}
\begin{enumerate}
  \item
    First entry.
  \item
    Second entry.
\end{enumerate}
    \end{tcblisting}
    
  \item
    One can use different settings for different levels of nestedness:
    \begin{tcblisting}{title={Settings depending on level}}
\begin{enumerate}[label = \Roman*]
  \item
    First entry.
    \begin{enumerate}[label = \alph*]
      \item
        First entry, first subentry.
      \item
        First entry, second subentry.
    \end{enumerate}
  \item
    Second entry.
    \begin{enumerate}[label = \arabic*]
      \item
        Second entry, first subentry.
      \item
        Second entry, second subentry.
    \end{enumerate}
\end{enumerate}
    \end{tcblisting}
    One can also use different global settings for different depths:
    \begin{tcblisting}{title={Global settings depending on level}}
\setlist[enumerate, 1]{label = (\roman*)}
\setlist[enumerate, 2]{label = (\alph*)}
\begin{enumerate}
  \item
    First entry.
    \begin{enumerate}
      \item
        First entry, first subentry.
      \item
        First entry, second subentry.
    \end{enumerate}
  \item
    Second entry.
    \begin{enumerate}
      \item
        Second entry, first subentry.
      \item
        Second entry, second subentry.
    \end{enumerate}
\end{enumerate}
    \end{tcblisting}
    The counter of the first depth and second depth can be accessed via \texttt{enumi} and \texttt{enumii}: 
    \begin{tcblisting}{title={Accessing level counters}}
\setlist[enumerate, 1]{label = (\arabic*)}
\setlist[enumerate, 2]{label = (\arabic{enumi}.\alph*)}
\begin{enumerate}
  \item
    An entry.
    \begin{enumerate}
      \item
        Again an entry.
      \item
        Again an entry.
    \end{enumerate}
  \item
    Another entry.
    \begin{enumerate}
      \item
        Yet another entry.
      \item
        Yet another entry.
    \end{enumerate}
\end{enumerate}
    \end{tcblisting}
    
  \item
    One often uses a certain kind of list environment multiple times with a specific formatting in the same way.
    In this case it is best to created a cloned version of this list environment, and then set global settings for this list.
    \begin{tcblisting}{title={Cloning lists}}
% clone enumerate as equivalenceslist, allowing up to 2 levels
\newlist{equivalenceslist}{enumerate}{2}
% set the formatting
\setlist[equivalenceslist,1]{label = (\roman*)}
\setlist[equivalenceslist,2]{label = (\alph*), leftmargin = *}
% an example
Let $M$ be an $R$-module.
For every collection of elements $x_1, \dotsc, x_n \in M$ the following conditions are equivalent:
\begin{equivalenceslist}
  \item
    For every $R$-module $N$ and every choice of elements $y_1, \dotsc, y_n \in N$ there exists a unique module homomorphism $f \colon M \to N$ with $f(x_i) = y_i$ for every $i = 1, \dotsc, n$.
  \item
    The elements $x_1, \dotsc, x_n$ are a basis of $M$, i.e.
    \begin{equivalenceslist}
      \item
        the elements $x_1, \dotsc, x_n$ are linearly independent, and
      \item
        the elements $x_1, \dotsc, x_n$ are a generating set of $M$.
    \end{equivalenceslist}
\end{equivalenceslist}
    \end{tcblisting}
\end{itemize}
